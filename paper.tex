%
\documentclass[a4paper,12pt]{memoir}
\usepackage{msiu_term_work}
\usepackage{floatflt}

\lstset{language=Ruby,inputencoding=utf8/koi8-r,basicstyle=\small,
stringstyle=\ttfamily,xleftmargin=1cm}

\begin{document}

\renewcommand{\contentsname}{{\Large{Содержание}\hfill}}

\title{Методы хранения и обработки информации}
{Вычисление суммы углов, под которыми рёбра выпуклой оболочки пересекают заданный отрезок.
Вычисление суммы площадей проекций граней, центр и все вершины которых находятся от плоскости $\mathit x = 2$ на расстоянии строго меньше 1}
{2362}
{А.\+О.~Дубинин}
{к.ф.-м.н., доцент}
{Е.\+А.~Роганов}
{2014}

\section{Введение}

Проект «Выпуклая оболочка»\cite{convex} решает задачу индуктивного 
перевычисления 
выпуклой оболочки последовательно поступающих точек плоскости и таких её
характеристик, как периметр и площади выпуклой оболочки. Целью данной работы является
вычисление количества вершин выпуклой оболочки, которые находятся в заданном треугольнике или в единичной окрестности заданного треугольника.
Решение этой задачи требует знания теории индуктивных
функций~\cite{roganov-2002}, основ аналитической геометрии, векторной алгебры
и языка Ruby~\cite{ruby}.

Прoект «Изображение проекции полиэдра»~\cite{polyedr}~--- пример
классической задачи, для успешного решения которой необходимо знакомство с
основами вычислительной геометрии. Задачей, решаемой в данной работе, является
модификация эталонного проекта с целью определения суммы длин проекций невидимых 
частей частично видимых рёбер, образующих с горизонтальной плоскостью угол не более $\pi/7$, центр которых находится строго внутри сферы $x^2+y^2+z^2=4$. 
Для этого необходимы хорошее понимание ряда разделов 
аналитической геометрии и векторной алгебры, основ объектно-ориентированного
программирования и языка Ruby. 

Общее количество строк в рассмотренных проектах составляет около $1180$, из кoтoрых
бoлее $350$ были изменены или дoбавлены автoрoм в прoцессе рабoты
над задачами мoдификации.

\section{Модификация проекта «Выпуклая оболочка»}

\subsection{Постановка задачи}

Модифицируйте эталонный проект таким образом, чтобы вычислялось количество пар вершин выпуклой оболочки, расстояние между которыми не превосходит единицу.
На начальном этапе работы программа запрашивает координаты вершин выпуклой оболочки, которая впоследствии индуктивно добавляет ее в выпуклую оболочку. Для решения нашей задачи необходимо выводить, помимо значений периметра и площади, количество пар вершин выпуклой оболочки, расстояние между которыми не превосходит единицу.

Для выполнения задания необходимо обладать базовыми знаниями в линейной алгебре. 2 точки, при их соединении, образуют отрезок, который можно представить в виде вектора. Чтобы получить длину (модуль) вектора $\overrightarrow{AB}$ необходимо воспользоваться следующей формулой:
$$\mid \overrightarrow{AB} \mid = \sqrt{(B_{x}-A_{x})^2+(B_{y}-A_{y})^2}$$  

Рассмотрим пример. Для последовательности точек $A(0, 0)$, $B(1, 0)$ и $C(0, 5)$ программа
выводит 1 в качестве количествa пар, расстояние между которыми не превосходит единицу.
После того,как мы вели  следующую точку $D(1, 5)$, количество ребер, расстояние между которыми не превосходит единицу, равняется $2$ (рис.~1).

\begin{figure}[ht!]
\begin{center}
\includegraphics[scale=0.7]{images/picture1}
\center{\texttt{Рис.~1.}}
\end{center}
\end{figure}
\newpage

\sloppy Все требуемые функции будут вписываться в файлы проекта \texttt{convex.rb} и \texttt{r2point.rb}.

\subsection{Решение}

Для выполнения задания необходимо вычислять растояние между вершинами выпуклой оболочки. В эталонном проекте изначально присутствует функция \texttt{dist}, которая считает расстояние между вершинами выпуклой оболочки.
\subsection{Модификация кода}
При реализации кода были произведены изменения в четырех файлах эталонного проекта: \texttt{r2point.rb}, \texttt{convex.rb}, \texttt{run\_convex.rb}, \texttt{run\_tkconvex.rb}.
Рассмотрим  изменения кода в файле \texttt{r2point.rb}. В него был добавлен метод \texttt{distance}:

\begin{small}
\begin{verbatim}
class R2Point
...
  def distance(a)
    return (self.dist(a)<=1)? 1 : 0
  end
  ...
\end{verbatim}
\end{small}

Данный метод рассматривает две точки, и проверяет, не превышает ли расстояние между вершинами $1$. Для этого используется метод \texttt{dist} класса \texttt{R2Point}  В случае, если расстояние меньше либо равно единице, то метод \texttt{distance} возвращает значение 1, в противном случае, возвращается 0. На этом модификация файла \texttt{r2point.rb} завершена.
\newpage
Рассмотрим модификацию файла \texttt{convex.rb}. В классе \texttt{Segment} добавляется метод \texttt{dist}:

\begin{small}
\begin{verbatim}
class Segment < Figure
  ...
  def dist
    return (@p.dist(@q) <= 1)? 1 : 0
  end
  ...
\end{verbatim}
\end{small}

Этот метод проверяет длину двуугольника (отрезка). Если она больше единицы возвращается значение 0, иначе возвращается значение 1.

Затем редактируется класс \texttt{Polygon}. Первоначально при создании объекта класса \texttt{Polygon} создается треугольник. Необходимо при инициализации объекта посчитать расстояния между вершинами треугольника и узнать, сколько пар вершин, между которыми расстояние не более единицы.

\begin{small}
\begin{verbatim}
class Polygon < Figure
  attr_reader :points, :perimeter, :area

  def initialize(a, b, c)
    ...
    @dist = a.distance(b) + b.distance(c) + c.distance(a)
  end
  ...
\end{verbatim}
\end{small}

В последстви, при добавлении новых точек индуктивно вычисляется расстояние от добавляемой точки, до первой и последней вершин выпуклой облочки :

\begin{small}
\begin{verbatim}
  ...
  @perimeter += t.dist(@points.first) + t.dist(@points.last)
  @dist += t.distance(@points.first) + t.distance(@points.last)
  @points.push_first(t)
  ...
\end{verbatim}
\end{small}

Как и при добавлени, так и при удалении вычисляется расстояние между вершинами, и если расстояние между парами вершин превышает $1$, то удаляются из переменной.

\begin{small}
\begin{verbatim}
 p = @points.pop_first
   while t.light?(p, @points.first)
     @perimeter -= p.dist(@points.first)
     @area      += R2Point.area(t, p, @points.first).abs
     @dist      -= p.distance(@points.first)
     p  = @points.pop_first
   end
 @points.push_first(p)
  ...
  p = @points.pop_last
    while t.light?(@points.last, p)
      @perimeter -= p.dist(@points.last)
      @area      += R2Point.area(t, p, @points.last).abs
      @dist      -= p.distance(@points.last)
      p = @points.pop_last
    end
  @points.push_last(p)
\end{verbatim}
\end{small}

Далее расмотрим примеры работы программы. 
В случае, когда создается объект класса \texttt{Void} количество вершин выпуклой оболочки равно нулю. Когда добавляется вершина создается объект класса \texttt{Point}. В обоих случаях количество вершин оболочки меньше двух, считать расстояние между вершинами невозможно и возвращается число 0. При создании объектов классов \texttt{Segment} и \texttt{Polygon} количество вершин больше либо равно двум и необходимо считать расстояния между вершинами.
Для точек  с координатами$(0, 0)$ и $(1, 0)$ расстояние между которыми равно 1, следовательно, программа выведет 1 (рис.~2).

\begin{figure}[ht!]
\begin{center}
\includegraphics[scale=1.0]{images/picture2.png}
\center{\texttt{Рис.~2.}}
\end{center}
\end{figure}
\newpage
При добавелнии новой точки $(0, 1)$, расстояние между всеми точками пересчитывается, и программа выводит результат 2. Так как растояние между вершинами $(0, 0)$ и $(0, 1)$, $(0, 0)$ и $(1, 0)$ равно 1, а расстояние между вершинами $(0, 1)$ и $(1, 0)$ превосходит 1 (рис.~3).

\begin{figure}[ht!]
\begin{center}
\includegraphics[scale=1.0]{images/picture3.png}
\center{\texttt{Рис.~3.}}
\end{center}
\end{figure}


При добавлении точки $(1, 1)$, программа пересчитывает расстояния между точками и выводит ответ $4$ (рис.~4).

\begin{figure}[ht!]
\begin{center}
\includegraphics[scale=1.0]{images/picture4.png}
\center{\texttt{Рис.~4.}}
\end{center}
\end{figure}

\newpage
Далее добавляется новая точка $(1, 3)$. Точка $(1, 1)$ удаляется и выводится ответ 2 (рис.~5).

\begin{figure}[ht!]
\begin{center}
\includegraphics[scale=1.0]{images/picture5.png}
\center{\texttt{Рис.~5.}}
\end{center}
\end{figure}


\section{Модификация проекта «Изображение проекции полиэдра»}

\subsection*{Точная постановка задачи}
Назовём точку в пространстве «хорошей», если она находится на расстоянии строго меньше 1 от плоскости $\mathit x =2$ Модифицируйте эталонный проект таким образом, чтобы определялась и печаталась следующая характеристика полиэдра: сумма площадей проекций граней, центр и все вершины которых — «хорошие» точки.


\subsection*{Решение данной задачи}
Для выполнения поставленной задачи необходимо модифицировать код в файле $\texttt{common/polyedr.rb}$ 
Методы проверки центра грани, всех ее вершин и вычисление площади проекции грани выполняется в методе $\texttt{draw}$ класса $\texttt{Polyedr}$ в файле $\texttt{shadow/polyedr.rb}$:


\begin{small}
\begin{verbatim}
class Polyedr 
  attr_reader :edges, :facets, :total_area
  def initialize(file)
    @total_area = 0.0
    ...
    nf.times do
      ...
     @total_area += facet.area if facet.all_good?() 
      end
    puts "Нужная сумма: #{@total_area}"
    end
  end
end
\end{verbatim}
\end{small}

 \subsection*{Модификация кода}

В файле $\texttt{common/polyedr.rb}$ добавим метод $\texttt{is\_good?}$, который позволит проверять точку на расстояние от плоскости $\mathit x=2$. Расстояние от точки $\mathit{M(M_{x},~M_{y},~M_{z})}$ для плоскости, заданной уравнением $\mathit{Ax+By+Cz+D=0},$ от точки вычисляется по формуле: 
$$ \mathit{d=\frac{|AM_{x}+BM_{y}+CM_{z}+D|}{\sqrt{A^2+B^2+C^2}}}.$$
Так как в условии задачи плоскость задана уравнением $\mathit x=2$, то уравнение принимает вид
$$ \mathit{d=\frac{|M_{x}+0+0-2|}{\sqrt{1+0+0}}=|M_{x}-2|}.$$
Если расстояние больше, либо равно единице, то метод вернет $\texttt{false}$, иначе, метод вернет $\texttt{true}$ :
\begin{small}
\begin{verbatim}
class R3 
  ...
  def is_good?()
    (x - 2).abs < 1
  end
end
\end{verbatim}
\end{small}

Затем добавим методы в класс $\texttt{Facet}$. Метод $\texttt{all\_good?}$ возвращает $\texttt{true}$, когда все вершины грани и ее центр ~--- хорошие точки, и $\texttt{false}$, когда не все. Координаты центра грани находятся с помощью метода $\texttt{center}$, который используется в файле $\texttt{shadow/polyedr.rb}$ эталонного проекта.
\begin{small}
\begin{verbatim}
class Facet 
  ...
  def all_good?()
    @vertexes.all?{|i| i.is_good?()} and center.is_good?()
  end
  ...
end
\end{verbatim}
\end{small}

Далее напишем вспомогательный метод $\texttt{triangle\_area(a, b, c)}$, который считает площадь треугольника, заданного вершинами $\texttt{a, b}$ и $\texttt{c}$. В данном методе используется формула вычисления площади треугольника по вершинам при помощи определителя матрицы:

$$ \mathit S = \frac{1}{2}\begin{vmatrix}
a_{x}-c_{x} & a_{y}-c_{y} \\ 
b_{x} - c_{x} & b_{y} - c_{y}
\end{vmatrix} .$$

\begin{small}
\begin{verbatim}
class Facet
  ...
  def triangle_area(a, b, c)
    0.5*((a.x-c.x)*(b.y-c.y)-(a.y-c.y)*(b.x-c.x)).abs
  end
end
\end{verbatim}
\end{small}

Затем реализуем метод $\texttt{area}$, который будет основным для подсчета площади проекции грани. Данный метод будет базироваться на вспомогательном методе $\texttt{triangle\_area(a, b, c)}$, где в качестве агрумента $\texttt{с}$ будет передаваться центр грани. Циклом пройдем по всем вершинам и просуммируем площади всех треугольников, из которых состоит грань, тем самым получим площадь проекции грани.

\begin{small}
\begin{verbatim}
class Facet
  ...
  def area
    area = 0.0; c = center
    (-1...@vertexes.size - 1).each do |i|
      area += triangle_area(@vertexes[i], @vertexes[i + 1], c)
    end
    area
  end
  ...
end
\end{verbatim}
\end{small}

На этом решение поставленной завершено. Пример работы программы 
с модифицированным файлом $\texttt{test3.geom}$ можно увидеть на рис.2 и его содержание представленно ниже.
\begin{figure}[ht!]
\begin{center}
\includegraphics[width=0.8\hsize]{images/shadow}
\end{center}
\caption{Работа программы «Изображение проекции полиэдра»}
\end{figure}
\newpage\begin{small}
\begin{verbatim}
1	0	0	0
10 3 12
2.0 0 0
2.0 2 0
1.5 0 2
1.5 2 2
1.1 0 0
1.1 2 0
0 0 0
0 2 0
1.5 0 2
1.5 2 2
4 1 2 4 3
4 5 6 4 3
4 7 8 4 3

\end{verbatim}
\end{small}




\begin{thebibliography}{}

\bibitem{convex}
\link{http://edu.msiu.ru//files/25029-lecture.html}~---
Описание проекта «Выпуклая оболочка».

\bibitem{roganov-2002}
Е.А. Роганов
{\em Основы информатики и программирования.}~---
М., МГИУ, 2002.

\bibitem{ruby}
\link{http://ru.wikipedia.org/wiki/Ruby}~---
Википедия (свободная энциклопедия) о языке Ruby.

\bibitem{polyedr}
\link{http://edu.msiu.ru/files/26490-lecture.html}, 

\link{http://edu.msiu.ru/files/26929-lecture.html}~---
Описание проекта «Изображение проекции полиэдра».

\end{thebibliography}

\newpage

\end{document}
