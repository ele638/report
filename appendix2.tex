\section{Приложение Б}

Модификация эталонного проекта \texttt{common/polyedr.rb}:
\begin{small}
\begin{verbatim}
 
>   def good?(coef=1.0) #проверка точки на "хорошесть" (добалвено)
>    if (2.0 - @x).abs < 1.0 #если расстояние до прямой x=2 строго меньше 1, то вернется true, иначе false
>      return true
>    else
>      return false
>    end 
>  end

>   def center
>    xc=(@fin.x+@beg.x)/2.0 
>    yc=(@fin.y+@beg.y)/2.0
>    zc=(@fin.z+@beg.z)/2.0
>    return R3.new(xc,yc,zc)
>  end

>  def func 
>    sum=0
>    edges.each do |e|
>      if e.center.good?(e.coef) && (e.beg.good?(e.coef) ^ e.fin.good?(e.coef))
>        sum += e.length
>      end
>    end
>    return sum
>  end
\end{verbatim}
\end{small}

\newpage
Модификация эталонного проекта \texttt{shadow/run\_polyedr.rb}:

\begin{small}
\begin{verbatim}
> #!/usr/bin/env ruby
> # encoding: UTF-8
> require './polyedr'
> require '../common/tk_drawer'
> TkDrawer.create
> %w(test1 ).each do |name|
>   puts '============================================================='
>   puts "Начало работы с полиэдром '#{name}'"
>   start_time = Time.now
>   a=Polyedr.new("../data/#{name}.geom")
>   a.draw
>   puts a.func
>   puts "Изображение полиэдра '#{name}' заняло #{Time.now - start_time} сек."
>   print 'Hit "Return" to continue -> '
>   gets
> end
\end{verbatim}
\end{small}
