\section{Введение}

Проект «Выпуклая оболочка»\cite{convex} решает задачу индуктивного 
перевычисления 
выпуклой оболочки последовательно поступающих точек плоскости и таких её
характеристик, как периметр и площади выпуклой оболочки. Целью данной работы является
вычисление суммы углов, под которыми рёбра выпуклой оболочки пересекают заданный отрезок.
Решение этой задачи требует знания теории индуктивных
функций~\cite{roganov-2002}, основ аналитической геометрии, векторной алгебры
и языка Ruby~\cite{ruby}.

Прoект «Изображение проекции полиэдра»~\cite{polyedr}~--- пример
классической задачи, для успешного решения которой необходимо знакомство с
основами вычислительной геометрии. Задачей, решаемой в данной работе, является
модификация эталонного проекта с целью Вычисление суммы площадей проекций граней, центр и все вершины которых находятся от плоскости $\mathit x = 2$ на расстоянии строго меньше 1. 
Для этого необходимы хорошее понимание ряда разделов 
аналитической геометрии и векторной алгебры, основ объектно-ориентированного
программирования и языка Ruby. 

Общее количество строк в рассмотренных проектах составляет около 630, из кoтoрых
бoлее 80 были изменены или дoбавлены автoрoм в прoцессе рабoты
над задачами мoдификации.
