\section{Введение}

Проект «Выпуклая оболочка»\cite{convex} решает задачу индуктивного 
перевычисления 
выпуклой оболочки последовательно поступающих точек плоскости и таких её
характеристик, как периметр и площадь. Целью данной работы является
определение количества вершин выпуклой оболочки, попадающих в 1-окрестность
заданного треугольника. Решение этой задачи требует знания теории индуктивных
функций~\cite{roganov-2002}, основ аналитической геометрии и векторной алгебры
и языка Ruby~\cite{ruby}.

Проект «Изображение проекции полиэдра»~\cite{polyedr}~--- пример
классической задачи, для успешного решения которой необходимо знакомство с
основами вычислительной геометрии. Задачей, решаемой в данной работе, является
модификация эталонного проекта с целью определения суммы длин проекций 
невидимых частей частично видимых рёбер заданного полиэдра. Для этого 
необходимы хорошее понимание ряда разделов аналитической геометрии и 
векторной алгебры, основ объектно-ориентированного
программирования и языка Ruby. 
  
Для подготовки пояснительной записки необходимо знакомство с программой
компьютерной вёрстки \LaTeX~\cite{rlatex}, умение набирать математические 
формулы~\cite{texbook} и включать в документ графические изображения и исходные
коды программ.

Общее количество строк в рассмотренных проектах составляет около 302, из которых
более 60 были изменены или добавлены автором в процессе работы
над задачами модификации.
