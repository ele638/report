\section{Введение}
Проект «Выпуклая оболочка»\cite{convex} решает задачу индуктивного 
перевычисления 
выпуклой оболочки последовательно поступающих точек плоскости и таких её
характеристик, как периметр и площади выпуклой оболочки. Решение этой задачи требует знания теории индуктивных
функций~\cite{roganov-2002}, основ аналитической геометрии, векторной алгебры
и языка Ruby~\cite{ruby}.

Прoект «Изображение проекции полиэдра»~\cite{polyedr}~--- пример
классической задачи, для успешного решения которой необходимо знакомство с
основами вычислительной геометрии. Для этого необходимы хорошее понимание ряда разделов 
аналитической геометрии и векторной алгебры, основ объектно-ориентированного
программирования и языка Ruby. 

В данной курсовой работе рассматриваются модификации двух эталонных про-
ектов «Выпуклая оболочка» и «Изображение проекции полиэдра».
Целями работы являются:
\begin{enumerate}[1)]
\item в проекте «Выпуклая оболочка» вычислить количество пар вершин выпуклой оболочки, расстояние между которыми не превосходит единицу; 
\item в проекте «Изображение проекции полиэдра»  вычислить
сумму длин рёбер, середина и ровно один из концов которых находятся
на расстоянии строго меньше 1 от плоскости $\mathit x = 2$.

\end{enumerate}
Общее количество строк в рассмотренных проектах составляет около $000000000000$, из кoтoрых
бoлее $0000000000$ были изменены или дoбавлены автoрoм в прoцессе рабoты
над задачами мoдификации.