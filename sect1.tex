\section{Модификация проекта «Выпуклая оболочка»}

  Модифицируйте эталонный проект таким образом, чтобы вычислялось количество
вершин, лежащих в 1-окрестности заданного заполненного треугольника.

  После запуска программа предлагает пользователю ввести шесть координат трех 
точек треугольника, затем последовательно вводятся координаты вершин 
выпуклой оболочки. Введенная точка индуктивно добавляется в выпуклую оболочку. Нам 
же необходимо вместе со значениями периметра и площади выводить количество 
точек, лежащих в 1-окрестности заданного заполненного треугольника.
\subsection{Решение}
  Данное решение можно разложить на три типа положения вершин: вершина 
лежит в заданном треугольнике; вершина лежит в единичной окрестности заданного 
треугольника; точка лежит вне единичной окрестности заданного треугольника.
\subsection{Тип №1}
  Точка находится

